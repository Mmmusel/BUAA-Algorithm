\documentclass{article}

% Language setting
% Replace `English' with e.g. `Spanish' to change the document language
\usepackage[english]{babel}


% Set page size and margins
% Replace `letter paper' with`a4paper' for UK/EU standard size
\usepackage[a4paper,top=1.5cm,bottom=1.5cm,left=2cm,right=2cm,marginparwidth=1.75cm]{geometry}
\usepackage[UTF8,adobefonts]{ctex}
% Useful packages
\usepackage{amssymb,amsmath}
\usepackage{graphicx}
\usepackage{listings}
\usepackage[ruled]{algorithm2e}
\usepackage[colorlinks=true, allcolors=blue]{hyperref}

\SetKwProg{Function}{function}{:}{end}


\title{ 计算机学院《算法设计与分析》第四次作业}
\author{20373363 李子涵}

\begin{document}
\maketitle

\section{P、NP、NPC问题} % TODO 作业题目1名字

% 作业题目1描述
对下面的每个描述,请判断其是正确或错误,或无法判断正误。对于你判为错误$/$无法判断的描述,请说明它为什么是错误$/$无法判断的。

1. $P $类问题为$ NP $类问题的真子集。

无法判定。P问题在多项式时间内可解,所以也多项式时间内可验证,$P\subseteq NP$。但是$NP \subseteq P$无法判断,NP是否等于P还是开放性问题。

2. 如果假设$ P \neq NP$,则$ NP $完全问题可以在多项式时间内求解。

错误,如果$ NP $完全问题可以在多项式时间内求解,则所有$ NP $问题可以在多项式时间内求解,与假设$ P \neq NP $矛盾。

3. 若$ SAT $问题可以用复杂度为$ O(n^9) $的算法来解决,则所有的$ NP $完全问题都可以在多项式时间内被解决;

正确。$SAT$属于$NP$完全问题,如果$SAT$问题可以在多项式时间解决,则存在$NP$完全问题多项式时间可解,则所有$NP$完全问题为多项式可解。

4. 对于一个$ NP $完全问题,其所有种类的输入均需要用指数级的时间求解。

错误。虽然$SAT$是$NP$完全问题,但是对于所有满足$2-SAT$条件的输入,都可以在多项式时间内求解。

\section{颜色交错最短路问题} % TODO 作业题目2名字

% TODO 作业题目2描述
给定一个无权有向图$ G =< V, E >$(所有边长度为 1),其中$ V = \{v_0, v_1,...v_{n−1}\}$,且这个图中的每条边不是红色就是蓝色$(\forall e \in E, e.color = red $或$ e.color = blue)$,图$ G $中可能存在自环或平行边。

现给定图中两点$ v_x, v_y$,请设计算法求出一条从$ v_x $到$ v_y$,且红色和蓝色边交替出现的最短路径。如果不存在这样的路径,则输出-1。请给出分析过程、伪代码以及算法复杂度。

\subsection{分析过程} % TODO 作业题目2策略

求无权有向图的最短路径,考虑广度优先搜索$BFS$算法。此题相对于基础BFS算法的区别在于:

1.判定点是否已被遍历时,对于与前驱节点的不同边的颜色,需要分别考虑。故将点标记为\textbf{已遍历}并记录父节点信息时,需要按照边的\textbf{两种颜色分别记录}。

2.判定后继节点是否可以被搜索的条件,除了此后继节点未被遍历,还需要考虑是否与上一条边满足颜色交错。所以每个点入队时,还需要额外记录节点与其前驱节点的边的颜色。此外,由于从起点出发时没有前驱节点及入边,故可以选择红色和蓝色两种颜色的边,所以初始化队列时需要两次加入起点,标记不同颜色。

由于广度优先搜索的队列性质,可使用贪心策略:第一次搜索到终点$v_y$且与其父节点符合颜色交错条件时,即可返回此最短路径;当广度优先搜索结束且未返回路径时,说明不存在这样的颜色交错路径,返回-1。



\subsection{伪代码} % 作业题目2伪代码

% TODO 伪代码模板

\begin{algorithm}[H]

\caption{颜色交错最短路问题的广度优先算法}
\LinesNumbered
\KwIn{无权有向图$G(V,E)$,起点$v_x$,终点$v_y$}
\KwOut{$v_x$到$v_y$的最短路径}

\Function{$shortestPath(V,E,v_x,v_y)$}{
	\tcp{\mbox{分别初始化红边和蓝边指向的节点的三个辅助数组,红色边指向的下标为0,蓝色边指向的下标为1}}
	\For{$v \in V$}{
		\tcp{\mbox{遍历状态数组}$has[\ ]$\mbox{为0表示未遍历,1表示已遍历}}
		$has[v][0],has[v][1] \leftarrow 0,0$\;
		$pred[v][0],pred[v][1] \leftarrow NULL,NULL$\;
		$dist[v][0],dist[v][1] \leftarrow \infty,\infty$\;
	}
	$has[v_x][0],has[v_x][1],pred[v_x][0],pred[v_x][1],dist[v_x][0],dist[v_x][1] \leftarrow 1,1,-1,-1,0,0$\;
	\tcp{\mbox{源点入队}}
	$Q.Enqueue((v_x,0)),Q.Enqueue((v_x,1))$\;
	\While{等待队列$Q$不为空}{
		\tcp{$(u,lastColor)$\mbox{当前处理顶点u及其指向u的边的颜色}}
		$(u,lastColor) \leftarrow Q.Dequeue()$\;
		\For{$v \in G.Adj[u]$}{
			\tcp{$nextColor$\mbox{是u到v的边的颜色}}
			$nextColor \leftarrow\ (<u,v>.color\ ==\ red)\ {\bf ?}\ 0\ {\bf :}\ 1$\;
			\tcp{\mbox{成功搜索条件:}顶点没有被$nextColor$的边检索过、检索符合颜色交错条件}
			\If{$has[v][nextColor]==0\ {\bf and}\ nextColor\neq lastColor$}{ 
				$has[v][nextColor] \leftarrow 1$\;
				$pred[v][nextColor] \leftarrow u$\;
				$dist[v][nextColor] \leftarrow dist[u][lastColor]+1$\;
				
				\tcp{\mbox{贪心策略:检索到终点返回,否则当前节点继续入队}}
				\If{$v==v_y$}{ 
				
					\Return{$minPath(pred,v,nextColor),dist[v][nextColor]$}\;
				}
				$Q.Enqueue((v,nextColor))$\;
			}
		}
	}

    \Return{$-1$}\;
}

\end{algorithm}


\begin{algorithm}[H]

\caption{输出最短颜色交错路径}
\LinesNumbered
\KwIn{父节点数组$pred[\ ][2]$,起点$v_x$,终点$v_y$,指向终点边的颜色$color$}
\KwOut{$v_x$到$v_y$的最短路径数组$path[\ ]$}

\Function{$minPath(pred[\ ][2],v_x,v_y,color)$}{
	\textbf{let }$path[\ ]$\textbf{ be new array}\;
	
	$v\leftarrow v_y$\;
	$path[0]\leftarrow v$\;
	$path.length\leftarrow 1$\;

	\While{$pred[v][color]\neq -1$}{
		
		$v\leftarrow pred[v][color]$\;
		$color\leftarrow 1-color$\;
		$path[path.length++]=v$\;
		
	}
	$path[path.length++]=v_x$\;
    \Return{$path.reverse()$}\;
}



\end{algorithm}

\subsection{时间复杂度分析}
状态个数$O(V)$,每次搜索新状态时枚举当前状态出发的有向边,总时间复杂度$T(|V|,|E|)=O(|V|+|E|)$。


\section{最小闭合子图问题}

对于一个有向图$ G =< V, E >$,其闭合子图是指一个顶点集为$ V' \subseteq V $的子图,且保证点集$ V'$中的所有出边都还指向该点集。换言之,$ V'$需满足对所有边 $(u, v) \in E$,如果点$ u $在集合$ V'$中,则点$ v $也一定在集合$ V'$中。

现给定一个包含$ n $个点的有向图$ G =< V, E >$,请设计算法求出该图中的闭合子图至少应包含几个顶点,并分析其时间复杂度。

例如,给定如下图所示的包含5个顶点的图,其闭合子图可能为:$\{v3\}, \{v0, v1, v2, v3, v4\}, \{v2, v4\}$。最小的闭合子图仅包含 1 个顶点,为$ {v3}$。请给出分析过程、伪代码以及算法复杂度。

\subsection{分析过程}
首先考虑简单情况,假设$ G $为有向无环图$(DAG)$,则必存在出度为 0 的点为闭合子图,且由贪心策略可知,此出度为0的点即为最小闭合子图。因为其他出度不为0的点,均需要将其后继节点加入此节点的闭合子图。

对于可能存在环路的有向图,对图$ G $计算强连通分量,使用强连通分量构造新的有向无环图$ G'$,将强连通分量作为$ G'$的节点,$E $中跨越两个强连通分量之间的边作为$ G'$的边。

新图$ G'$为有向无环图,只需要找到$ G'$中出度为 0 的节点,这些节点代表的强连通分量中,包含最少原图节点的即为最小闭合子图。

\subsection{伪代码}
\begin{algorithm}[H]
	\caption{最小闭合子图}
	\LinesNumbered
	\KwIn{有向图$G(V,E)$}	
	\KwOut{最小闭合子图的节点数}
	\Function{$minGraph($有向图$G(V,E))$}{
		$\{s_1,s_2,...,s_k\} \leftarrow scc(G)$\;
		$V'=\{s_1,s_2,...,s_k\}$\;
		$E'=\{<s_a,s_b>|<u,v> \in E,u\in s_a,v\in s_b\}$\;
		$ans=|V|$\;
		
		\For{$s_u \in V'$}{
			
			$out \leftarrow |\{<s_u,s_v>|<s_u,s_v>\in E'\}|$\;
			
			\If {$out==0$} {
				$ans \leftarrow min\{ans,|s_u|\}$\;
				
			}
		}
		\Return{$ans$}\;
	}

\end{algorithm}

\begin{algorithm}[H]
	\caption{计算强连通分量}
	\LinesNumbered
	\KwIn{有向图$G(V,E)$}	
	\KwOut{强连通分量}
	\Function{$scc(G)$}{
		
		$R\leftarrow \{\ \}$\;
		$G^R\leftarrow G.reverse()$\;
		$L\leftarrow DFS(G^R)$\;
		$color[1..V]\leftarrow WHITE$\;
		\For{$i \leftarrow L.length() \ ${\bf downto}$ \ 1$}{
			$u\leftarrow L[i]$\;	
			\If {$color[u]==WHITE$} {
				$L_{scc}\leftarrow DFS-Visit(G,u)$\;
				$R\leftarrow R\cup set(L_{scc})$\;
				
			}
		}
		\Return{$R$}\;
	}

\end{algorithm}



\begin{algorithm}[H]
	\caption{反向图执行DFS}
	\LinesNumbered
	\KwIn{有向图$G(V,E)$}	
	\KwOut{反向图执行DFS}
	\Function{$DFS(G)$}{
		\textbf{let }$color[1..|V|],L[1..|V|]$\textbf{ be new arrays}\;
		\For{$v\in V$}{
			$color[v]\leftarrow WHITE$\;	
		}
		\For{$v\in V$}{
			\If {$color[u]==WHITE$} {
				$L'\leftarrow DFS-Visit(G,v)$\;
				$L.append(L')$\;
				
			}
		}
		
		\Return{$L$}\;
	}

\end{algorithm}
\begin{algorithm}[H]
	\caption{DFS-Visit}
	\LinesNumbered
	\KwIn{有向图$G(V,E)$,顶点$v$}	
	\KwOut{按完成时刻从早到晚的顶点L}
	\Function{$DFS-Visit(G,v)$}{
		$color[v]\leftarrow GRAY$\;
		\For{$w\in G.Adj[v]$}{
			\If {$color[w]==WHITE$} {
				$L\leftarrow DFS-Visit(G,w)$\;
			}	
		}
		$color[v]\leftarrow BLACK$\;
		$L.append(v)$\;

		
		\Return{$L$}\;
	}

\end{algorithm}


\subsection{时间复杂度分析}
求解强连通分量复杂度$O(|V|+|E|)$,贪心策略遍历所有强连通分量复杂度$O(|V|)$,总时间复杂度$O(|V|+|E|)$。


\section{食物链问题}
给定一个食物网,包含$ n $个动物,$m $个捕食关系,第$ i $个捕食关系使用$ (s_i
, t_i) $表示,$s_t $捕食者,$t_i $表示被捕食者,根据生物学定义,食物网中不会存在环。

长度为$ k $的食物链指包含$ k $个动物的链:$a_1, a_2, ... , a_k$,其中$ a_i $会捕食$ a_{i+1}$,一个食物链为最大食物链当且仅当$ a_1 $不会被任何动物捕食,且$ a_k $不会捕食任何动物。

请设计一个高效算法计算食物网中最大食物链的数量,并给出分析过程、伪代码以及算法复杂度。

\subsection{分析过程}
$n $个动物作为节点$ V $,$m $个捕食关系作为边$ E$,构造有向无环图$ G(V, E)$。最大食物链可以表示为图上的一条路径。路径起点包含所有入度为 0的点,终点包含所有出度为 0的点。

求解路径数量可以使用深度优先搜索计算,注意到一个节点的所有后继捕食关系可能被前驱节点多次搜索,即存在重复的子问题,故考虑提高性能可以使用动态规划来缩减子问题规模。

易知捕食关系中不存在平行边,故考虑到达一个节点的所有路径数,只需要将所有前驱节点的路径数累加。动态规划设计如下:

1.状态设计:使用 dp[i] 表示编号为 i 的节点为路径终点的路径数量

2.状态转移:对于前驱节点$ pre[v] = \{u| < u, v >\in E, u \in V \},dp[v] = \sum_{u\in pre[v]}dp[u]$。

3.边界条件:所有路径起点(入度为0的点)的状态的路径数量为0

4.目标状态:所有路径终点(出度为0的点)的状态路径数目之和

再考虑遍历顺序,计算一个节点需要获取其所有前驱节点的状态,故遍历顺序使用图$G$的拓扑排序顺序即可。



\subsection{伪代码}

\begin{algorithm}[H]

	\caption{食物链问题——动态规划求解}
	\LinesNumbered
	\KwIn{动物$n$,捕食关系$m,(s_i,t_i)$}	
	\KwOut{食物链数量}
	\Function{$foodChain(n, m,(s_i,t_i))$}{
		$n$\textbf{个动物作为节点}$V,m$\textbf{个捕食关系}$(s_i,t_i)$\textbf{作为边}$E$\textbf{,构造有向无环图}$G(V, E)$\;
		
		$topo \leftarrow topoSort(G)$\;
		$ans \leftarrow 0$\;


		\For{$i \leftarrow 0 \ ${\bf to}$ \ n $} {
			$pre[topo[i]]\leftarrow  \{u| < u, topo[i] >\in E, u \in V \}$\;
			$out[topo[i]]\leftarrow  |\{u| < topo[i],u >\in E, u \in V \}|$\;
			$dp[topo[i]] \leftarrow \sum_{u\in pre[topo[i]]}dp[u]$\;

			\If {$out[topo[i]] == 0$} {
				$ans \leftarrow ans+dp[topo[i]]$\;
			}
			
		}

		\Return{$ans$}\;
	}

\end{algorithm}

\begin{algorithm}[H]

	\caption{拓扑排序}
	\LinesNumbered
	\KwIn{有向无环图$G(V,E)$}	
	\KwOut{顶点拓扑序}
	\Function{$topoSort(G)$}{
		\textbf{初始化空队列}$Q$\textbf{,拓扑排序结果数组}$ans$\;
		\For{$v \in V$}{
			\If{$v.in\_degree==0$}{
				$Q.Enqueue(v)$\;
			}
		}
		\While{$not Q.is\_empty()$}{
			$u\leftarrow Q.Dequeue()$\;
			$ans.append(u)$\;
\For{$v\in G.Adj(u)$}{
			$v.in\_degree\leftarrow v.in\_degree-1$\;
			\If{$v.in\_degree==0$}{
				$Q.Enqueue(v)$\;
			}
		}
		}
		

		\Return{$ans$}\;
	}

\end{algorithm}

\subsection{时间复杂度分析}
动态规划过程遍历每条边,复杂度$O(|E|)$,拓扑排序复杂度$O(|V|+|E|)$,所以总时间复杂度为$=O(m+n)$。

\section{景区限流问题问题}

已知某市有热门景区$ m $个,可以表示为集合$ S = \{s_1, s_2, ..., s_m\}$,有游客$ n $人,可以表示为$T = \{t_1, t_2..., t_n\}$。每名游客有$ k $个心仪的景区,但每个景区最多容纳 l 人,问最多有多少人能够去到自己心仪的景区,并给出分析过程、伪代码以及算法复杂度。

对于游客与景区的偏好关系,用$ H $表示,则$ H $可以表示如下形式,其中$ (t_u, s_v) $表示游客$t_u $偏好景点$ s_v$。


\subsection{分析过程}
这是一个二分图最大匹配问题,把每一个景区分成$l$个景区,即可采用匈牙利算法,依次检测左侧顶点,不断寻找交替路径进行增广:相邻景区顶点存在未匹配的景区顶点,直接匹配;如果相邻景区顶点已经匹配,则尝试寻找交替路径,增广成新匹配。
\subsection{伪代码}

\begin{algorithm}[H]

	\caption{景区限流问题——二分图匹配}
	\LinesNumbered
	\KwIn{游客$T$,景区$S$,偏好关系$H$,游客$t_u $偏好景点$ s_v$:$(t_u, s_v)$,每个景区最大游客量$l$}	
	\KwOut{到心仪景区游客的最大数量}
	\Function{$hungarian(T,S,H,l)$}{
		\textbf{游客}$T$,\textbf{景区}$S$,\textbf{偏好关系}$H$,\textbf{游客}$t_u$\textbf{偏好景点}$s_v:(t_u, s_v)$\textbf{构造二分图}$G(T,S',H')$\textbf{,其中每一个景区变成}$l$\textbf{个景区,同时游客对此景区的偏好也变成对这}$l$\textbf{个景区的偏好}\;
		\textbf{let }$matched[1..|S'|],color[1..|S'|]$\textbf{ be new arrays}\;
		\For {$u \in S'$} {
			$matched[u]\leftarrow NULL$\;
			
		}
		\For {$v \in T$} {
			\For {$u \in S$} {
				$color[u]\leftarrow WHITE$\;
			}
			$DFS-Find(G,color,matched,v)$\;
		}

		$ans\leftarrow 0$\;
		\For {$u \in S'$} {
				$ans\leftarrow ans+(matched[u]!=NULL)$\;
			}
		
		\Return{$ans$}\;
	}

\end{algorithm}

\begin{algorithm}[H]

	\caption{寻找交替路径}
	\LinesNumbered
	\KwIn{二分图$G(T,S,H)$,匹配数组$matched[\ ][l]$,辅助数组$color[\ ]$,顶点$v$}	
	\KwOut{是否存在从顶点$v$出发的交替路径}
	\Function{$DFS-Find(G,color,matched,v)$}{
		
		\For {$u \in G.Adj[v]$} {
			\If {$color[u] == BLACK$} {
				\textbf{continue}\;
			}
			$color[u] \leftarrow BLACK$\;
			\If {$matched[u]==NULL\ {\bf or}\ DFS-Find(G,matched(u))$} {
				$matched[u]\leftarrow v$\;
				\Return{$true$}\;
			}
			
			
		}
		
		
		\Return{$false$}\;
	}

\end{algorithm}

\subsection{时间复杂度分析}
二分图遍历游客顶点进行匹配,时间复杂度$O(|V|+|E|)$,其中$|V|=n+ml$,$|E|=nkl$,所以总时间复杂度为$O(n^2kl+nmkl^2)$。


\end{document}