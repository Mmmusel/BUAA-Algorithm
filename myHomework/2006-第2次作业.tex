\documentclass{article}

% Language setting
% Replace `English' with e.g. `Spanish' to change the document language
\usepackage[english]{babel}


% Set page size and margins
% Replace `letter paper' with`a4paper' for UK/EU standard size
\usepackage[a4paper,top=1.5cm,bottom=1.5cm,left=2cm,right=2cm,marginparwidth=1.75cm]{geometry}
\usepackage[UTF8,adobefonts]{ctex}
% Useful packages
\usepackage{amssymb,amsmath}
\usepackage{graphicx}
\usepackage{listings}
\usepackage[ruled]{algorithm2e}
\usepackage[colorlinks=true, allcolors=blue]{hyperref}

\SetKwProg{Function}{function}{:}{end}


\title{ 计算机学院《算法设计与分析》第二次作业}
\author{20373363 李子涵}

\begin{document}
\maketitle

\section{小跳蛙问题}

给定$n$块石头,依次编号为$1$到$n$,第$i$块石头的高度是$h_i$,青蛙最远跳跃距离$k$。

现有一只小跳蛙在第$1$块石头上,它重复以下操作,直到它到达第$n$块石头:

若它当前在第$i$块石头上,则可跳到第$j(i + 1\leq j\leq min(i + k, n))$块石头上,耗费的体力为$|h_i-h_j|$。

试设计算法求它最少耗费多少体力可以到达第$n$块石头,写出伪代码并分析算法的时间复杂度。

\subsection{状态设计}
$s[i]$表示跳到第$i$块石头上耗费的最小体力。

\subsection{状态转移}
青蛙当前所在第$i$块石头,青蛙的最远跳跃距离$k$,则青蛙可以从第$i-j$块石头跳过$j=1,2,.. ,min\{k,i-1\}$块石头到达第$i$块石头。此时耗费的体力为$s[i-j]+|h_i-h_{i-j}|$。

故转移方程为:$s[i]=\min\limits_{j\in [1..min(k,i-1)]}\{s[i-j]+|h_i-h_{i-j}|\}$


\subsection{边界条件}
小跳蛙在第一块石头,初始状态为$s[1]=0$。

\subsection{目标状态}
到达第$n$块石头的最小代价$s[n]$。



\subsection{伪代码}
\begin{algorithm}[H]

\caption{到达第$n$块石头耗费的最小体力}
\LinesNumbered
\KwIn{代表每块石头高度的数组$h[1.. n]$,青蛙的最大跳跃距离$k$}
\KwOut{到达第$n$块石头耗费的最小体力}

\Function{$minStrength(h[1.. n],k)$}{

	\tcp{\mbox{新建状态数组}$s[1..n]$}
	\textbf{let }$s[1..n]$\textbf{ be new array}\;
	
\tcp{\mbox{设置初始状态}}
	$s[1]\leftarrow 0$\;
	

	\For{$i\leftarrow{2}\ to\ n$}{

		\tcp{\mbox{赋初值为从前一块石头转移过来的体力}}
		$s[i]\leftarrow s[i-1]\ +\ abs(\ h[i]\ -\ h[i-1]\ )$\;
		
		
		\tcp{\mbox{记录跳到第$i$块石头时的最小体力}}
		\For{$j\leftarrow{2}\ to\ min(k,i-1)$}{
			
			$s[i]\leftarrow min\{s[i],s[j]\ +\ abs(\ h[i]\ -\ h[i-j]\ )\}$\;
				
			
		}
	}

	\tcp{\mbox{最小体力}}
	$minStrength\leftarrow s[n]$\;
	\Return {minStrength}\;
	
}

\end{algorithm}

\subsection{复杂度分析}
总状态是$O(n)$级别的,每个状态的转移是$O(k)$的,所以总时间复杂度为$T(n)=O(nk)$。

\section{二进制串变换问题}

给定两个长度均为$n$的仅由$0$和$1$组成的字符串$a$和$b$,你可以对串$a$进行如下操作:

1.对任意$ i, j(1 \leq i, j \leq  n)$,交换$a_i$和$a_j$,操作代价为$|i − j|$;

2.对任意$ i(1 \leq  i \leq  n)$,取反$a_i$,操作代价为$1$;

请你设计算法计算将串$a$变为串$b$所需的最小代价(只能对串$a$进行操作),写出伪代码并分析算法的时间复杂度。
 

\subsection{状态设计}
$c[i]$表示把串$a$的前$i$位变成串$b$的前$i$位的最小代价。

\subsection{状态转移}
由操作代价知,只有\textbf{连续两个}数字需要取反,且连续的两个数字\textbf{不相等}时,选择\textbf{交换}这两个数字的代价\textbf{严格小于}对两个数字分别取反。

故有如下递推式:
$$
c[i]=\begin{cases}
c[i-1]+1& {\bf if}\ (a[i]\neq b[i])\ {\bf and} \ {\bf (\ }(a[i-1]== b[i-1])\ {\bf or} \ (a[i-1]== a[i]){\bf \  )}\\
c[i-2]+1& {\bf if}\ (a[i]\neq b[i])\ {\bf and} \ (a[i-1]\neq b[i-1])\ {\bf and} \ (a[i-1]\neq a[i])\\
c[i-1]& {\bf if}\ a[i]==b[i]
\end{cases}
$$

\subsection{边界条件}
$i=0$时,串$a$和串$b$都是空串,不需要进行操作。$c[0]=0$。

$a[1]$为串$a$的第一个数字,不会与前序数字进行交换操作,只需考虑是否需要进行取反操作。如果$a[1]==b[1]$,第一个数字不需要取反,$c[1]=0$;否则$a[1]\neq b[1]$,第一个数字需要取反,$c[1]=1$。

\subsection{目标状态}
将串$a[1..n]$变为串$b[1..n]$所需的最小代价$c[n]$。

\subsection{伪代码}
\begin{algorithm}[H]

\caption{将串$a$变为串$b$所需的最小代价}
\LinesNumbered
\KwIn{代表串$a$的数组$a[1.. n]$,代表串$b$的数组$b[1.. n]$}
\KwOut{将串$a$变为串$b$所需的最小代价}

\Function{$minCost(a[1.. n],b[1.. n])$}{
	
	\tcp{\mbox{新建状态数组}$c[0..n]$}
	\textbf{let }$c[0..n]$\textbf{ be new array}\;
	
	\tcp{\mbox{设置初始状态}}
	$c[0]=0$\;
	$c[1]=\ (a[1]==b[1])\ {\bf ?}\ 0\ {\bf :}\ 1$\;
	
	
	\For{$i\leftarrow{2}\ to\ n$}{
		\If{$(a[i]==b[i])$}{
			$c[i]\leftarrow c[i-1]$\;
		}
		\ElseIf{$(a[i-1]== b[i-1])\ {\bf or} \ (a[i-1]== a[i])$}{
			$c[i]\leftarrow c[i-1]+1$\;
		}
		\Else {
			$c[i]\leftarrow c[i-2]+1$\;
		}
	}

	$minCost\leftarrow c[i]$\;
	\Return {minCost}\;
}

\end{algorithm}

\subsection{复杂度分析}
总状态是$O(n)$级别的,每个状态的转移是$O(1)$的,所以总时间复杂度为$T(n)=O(n)$。

\section{球队组建问题}

有$2n$个学生分为两排,每排有$n$个人,由左至右分别编号为$1, 2,.. , n$,如图所示。现在请你在这两排学生中挑选出一些学生组成一支球队,挑选出的学生编号必须是严格递增的(编号相同的两名学生最多只能取其中一个)。此外,为避免球队中的队员都来自同一排,不能同时选择同一排相邻的两名学生(例如,若选择第一排的$5$号同学,就不能再选择第一排的$4$号和$6$号同学)。组建队伍的总人数没有限制。

给出同学们的身高数据$h_{i,j},h_{1,k}(1\leq k\leq n)$表示第一排同学的身高,$h_{2,k}(1\leq k\leq n)$表示第二排同学的身高。请你设计算法使组建成的球队中队员的身高之和最大,写出伪代码并分析算法的时间复杂度。

\subsection{状态设计}
设按序号从小到大选择组建成的球队的队员。$sum[i][j]$表示选择第$i$排编号为$j$的的同学加入球队或者不选择编号为$j$的同学加入球队后,此时队伍的最大总身高。

注:


\noindent
\hangafter=1
\setlength{\hangindent}{48pt}
\hspace{45pt}1.\ $i=1$或$2$时,分别表示选择\textbf{第一排或第二排}编号$j$的同学加入球队。

\noindent
\hangafter=1
\setlength{\hangindent}{48pt}
\hspace{45pt}2.\ $i=0$时,表示\textbf{不选择}同学加入球队作为编号$j$的队员。


\subsection{状态转移}
考虑到\textbf{不能}同时选择同一排\textbf{相邻}的两名学生:
\begin{enumerate}
	\item 如果选择第一排编号为$j$的队员,则前一个编号为$j-1$的队员可以\textbf{不选}或者选择\textbf{第二排}编号为$j-1$的同学。此时可能获得的最大总身高为$h_{1,j}+max\ \{\ sum[0][j-1],\ sum[2][j-1]\ \}$。
	\item 如果选择第二排编号为$j$的队员,则前一个编号为$j-1$的队员可以\textbf{不选}或者选择\textbf{第一排}编号为$j-1$的同学。此时可能获得的最大总身高为$h_{2,j}+max\ \{\ sum[0][j-1],\ sum[1][j-1]\ \}$。
	\item 如果不选择编号$j$的同学加入球队作为编号$j$的队员,则前一个编号为$j-1$的队员可以选择\textbf{第一排或者第二排}编号为$j-1$的同学。此时可能获得的最大总身高为$max\ \{\ sum[1][j-1],\ sum[2][j-1]\ \}$。
\end{enumerate}

故有如下递推式:
$$
sum[i][j]=\begin{cases}
h_{1,j}+max\ \{\ sum[0][j-1],\ sum[2][j-1]\ \}& {\bf if}\ i=1\\
h_{2,j}+max\ \{\ sum[0][j-1],\ sum[1][j-1]\ \}& {\bf if}\ i=2\\
max\ \{\ sum[1][j-1],\ sum[2][j-1]\ \}& {\bf if}\ i=0
\end{cases}
$$

\subsection{记录决策方案}

\textbf{记录前序队员所在排数}:数组$row[i][j]$,表示第$j$个队员来自第$i$排时,第$j-1$个队员来自于第$row[i][j]$排,即$sum[i][j]$状态是由$sum[\ row[i][j]\ ][\ j-1\ ]$状态转移而来的。从而利用数组$row$迭代得到编号为$j,j-1,.. ,1$的队员来自第几排。

\textbf{记录最大身高和对应的决策方案}:数组$choose[1..choose.length]$,其中$choose[j]$记录每个编号为$j$的队员来自第几排$(choose[j]=1$或$2)$或者不选择编号为$j$的队员$(choose[j]=0)$。


状态转移结束后,找到$sum[k][n](k\in [1..3])$中最大身高和对应的下标$k$,记录$choose[n]=k$,就可以利用记录前序队员所在排数的$row[i][j]$数组,得到前序队员所在排数$row[\ choose[n]\ ][\ n-1\ ]$得到第$n-1$前序队员所在的排数,进而迭代得到所有前序队员所在的排数,记入数组$choose[1..choose.length]$,代表最大身高和对应的决策方案。



\subsection{边界条件}
编号为$j=1$的队员没有选择的前置约束,故对应的最大总身高$sum[0][1]=0,\ sum[1][1]=h_{1,1},\ sum[2][1]=h_{2,1}$,且第一个队员前没有队员,故$row[0][1],\ row[1][1],\ row[2][1]$无意义。

\subsection{目标状态}
则最大总身高对应的状态为$sum[k][n],\ s.t.\ sum[k][n]=max\{\ sum[0][n],\ sum[1][n],\ sum[2][n]\ \}$。对应的队员选择方案为数组$choose[1..choose.length]$,其中$choose[j]$记录每个编号为$j$的队员来自第几排$(choose[j]=1$或$2)$或者不选择编号为$j$的队员$(choose[j]=0)$。。



\subsection{伪代码}
\begin{algorithm}[H]

\caption{组建球队使得队员的身高之和最大,及队员的选择方案}
\LinesNumbered
\KwIn{代表队员身高的二维数组$h[1.. 2][1.. n]$,其中$h[i][j]$代表第$i$排编号为$j$的队员身高$h_{i,j}$}
\KwOut{组建的最多n个队员的球队的最大身高之和}

\Function{$maxSumHeight(h[1.. 2][1.. n])$}{
	
	\tcp{\mbox{新建状态数组和记录前序队员的数组}$sum[0..2][1..n],row[0..2][1..n]$}
	\textbf{let }$sum[0..2][1..n],row[0..2][1..n]$\textbf{ be new arrays}\;
	\tcp{\mbox{记录初始状态}}
	$sum[0][1]\leftarrow 0,\ sum[1][1]\leftarrow h[1][1],\ sum[2][1]\leftarrow h[2][1]$\;
	\tcp{\mbox{状态转移}}
	\For{$j\leftarrow{2}\ to\ n$}{
		\tcp{\mbox{第}$j$\mbox{个球员从第一排选择}}
		$sum[1][j]\leftarrow h[1][j]+ max\{\ sum[0][j-1],\ sum[2][j-1]\ \}$\;
		$row[1][j]\leftarrow\ (sum[0][j-1]\ >\ sum[2][j-1])\ {\bf ?}\ 0\ {\bf :}\ 2$\;
\tcp{\mbox{第}$j$\mbox{个球员从第一排选择}}
		$sum[2][j]\leftarrow h[2][j]+ max\{\ sum[0][j-1],\ sum[1][j-1]\ \}$\;
		$row[2][j]\leftarrow\ (sum[0][j-1]\ >\ sum[1][j-1])\ {\bf ?}\ 0\ {\bf :}\ 1$\;
\tcp{\mbox{不选择第}$j$\mbox{个球员}}
		$sum[0][j]\leftarrow max\{\ sum[1][j-1],\ sum[2][j-1]\ \}$\;
		$row[0][j]\leftarrow\ (sum[1][j-1]\ >\ sum[2][j-1])\ {\bf ?}\ 1\ {\bf :}\ 2$\;

	}
	\tcp{\mbox{记录最大身高和与其对应的第}$n$\mbox{个球员的排数}}
	$maxSumHeight\leftarrow 0,lastRow\leftarrow 0$\;

	\For{$k\leftarrow{0}\ to\ 2$}{
		\If{$sum[k][n]\ >maxSumHeight$}{
			$maxSumHeight\leftarrow sum[k][n],\ lastRow\leftarrow k$\;
		}
	}
\tcp{\mbox{新建记录决策方案数组}$choose[1..n]$}
	\textbf{let }$choose[1..n]$\textbf{ be new array}\;
	$choose[n]\leftarrow lastRow$\;
	\For{$j\leftarrow{n}\ to\ 2$}{
		$choose[j-1]\leftarrow row[choose[j]][j]$\;
	}

}

\end{algorithm}

\subsection{复杂度分析}
总状态是$O(n)$级别的,每个状态的转移是$O(1)$的,所以总时间复杂度为$T(n)=O(n)$。
\newpage
\section{括号匹配问题}

定义合法的括号串如下:

1.空串是合法的括号串;

2.若串$s$是合法的,则$(s)$和$[s]$也是合法的;

3.若串$a, b$均是合法的,则$ab$也是合法的。

现在给定由$'[',']'$和$'(',')'$构成的字符串,请你设计算法计算该串中合法的子序列的最大长度,写出伪代码并分析算法的时间复杂度。例如字串$([(])])$,最长的合法子序列$([()])$长度为$6$。

\subsection{状态设计}
$len[i][j](i\leq j)$表示给定序列中从$i$到$j$的子串中最长的合法子序列。



\subsection{状态转移}
分别考虑合法字符串的两种构造方式:

\textbf{对于构造方式1}:若串$s$是合法的,则$(s)$和$[s]$也是合法的;

对于子串$s[i.. j]$,如果$s[i]$和$s[j]$可以构成$(,)$或$[,]$,则由构造方式一推出的合法子序列的长度为$len[i][j]=2+len[i+1][j-1]$。

\textbf{对于构造方式2}:若串$a,b$均是合法的,则$ab$也是合法的。

对于子串$s[i.. j]$,存在$i\leq t<j,(t\in [i,j-1])$,使得子串$s[i.. j]$由子串$s[i.. t]$与子串$s[(t+1).. j]$拼接而成。构造方式二共有$(j-i)$种构造方式,对于每一个$t\in [i,j-1]$,推出的合法子序列的长度分别为$len[i][j]=len[i][t]+len[t+1][j]$。


故有如下递推式:
$$
len[i][j]=max\begin{cases}
2+len[i+1][j-1]& {\bf if}\ (s[i]+s[j])==\ '(,)'\ {\bf or}\ '[,]'\\
len[i][t]+len[t+1][j]& t\in [i,j-1]
\end{cases}
$$


\subsection{边界条件}
对于单个字符$s[i](i\in [1..n])$,都不能构成合法字符串,故初始状态为$len[i][i]=0$。

对于每个子串$s[i][i+1](i\in [1..n-1])$,初始状态为$$len[i][i+1]=\begin{cases}
2& {\bf if}\ s[i,i+1]==\ '(,)'\ {\bf or}\ s[i,i+1]==\ '[,]'\\
0& {\bf if}\ s[i,i+1]\neq \ '(,)'\ {\bf and}\ s[i,i+1]\neq \ '[,]'
\end{cases}
$$

\subsection{目标状态}
串$s[1..n]$中合法的子序列的最大长度$len[1][n]$。

\subsection{伪代码}
\begin{algorithm}[H]

\caption{串$s$中合法的子序列的最大长度}
\LinesNumbered
\KwIn{代表串$s$的字符串数组$s[1.. n]$}
\KwOut{串$s$中合法的子序列的最大长度}

\Function{$maxIllegalLength(s[1.. n])$}{
	
	\tcp{\mbox{新建状态数组}$len[0..n][0..n]$}
	\textbf{let }$len[0..n][0..n]$\textbf{ be new array}\;

	\tcp{\mbox{设置初始状态}}
	\For{$i\leftarrow{1}\ to\ n$}{
		$len[i][i]\leftarrow 0$\;
	}

	\For{$i\leftarrow{1}\ to\ n-1$}{
		
		\If{$s[i,i+1]==\ '(,)'\ or\ '[,]'$}{
			$len[i][i+1]\leftarrow 2$\;
		}
		\Else {
			$len[i][i+1]\leftarrow 0$\;
		}
	}
		\tcp{\mbox{状态转移}}
	\For{$k\leftarrow{2}\ to\ n-1$}{
		\For{$i\leftarrow{1}\ \ to\ \ {n-k}$}{

$j\leftarrow i+k$\;
\tcp{$k$\mbox{ 为子串长度,}$i$\mbox{ 为子串第一个字符下标,}$j$\mbox{为子串最后一个字符下标}}
	\tcp{\mbox{合法字符串构造方式一}}
			\If{$(s[i]+s[j])==\ '(,)'\ or\ (s[i]+s[j])==\ '[,]'$}{
				$len[i][j]\leftarrow 2+len[i+1][j-1]$\;	
			}
			\Else{
				$len[i][j]\leftarrow 0$\;
			}
\tcp{\mbox{合法字符串构造方式二}}
			\For{$t\leftarrow{i}\ to\ {j-1}$}{
$len[i][j]\rightarrow max\{\ len[i][j],\ len[i][t]+len[t+1][j]\ \}$\;
				
			}
		}
	}

	$maxIllegalLength\leftarrow len[1][n]$\;
\Return{maxIllegalLength}\;

}

\end{algorithm}

\subsection{复杂度分析}
总状态是$O(n^2)$级别的,每个状态的转移是$O(n)$的,所以总时间复杂度为$T(n)=O(n^3)$。

\newpage
\section{箱子问题}

给定$n$种箱子$a_1,.. ,a_n$,第$i$种箱子$a_i$可表示为$h_i\times w_i\times d_i$的长方体。请用这些箱子搭建一个尽可能高的塔:如果一个箱子$A$要水平的放在另一个箱子$B$上,那么要求箱子$A$底面的长和宽都严格小于箱子$B$。可以任意旋转箱子,每种箱子可以用任意次。

设计一个算法求出一个建塔方案使得该塔的高度最高,写出伪代码并分析算法的时间复杂度。


\subsection{状态设计}
(一)分析题意:
\begin{enumerate}
\item 对于一个$h_i\times w_i\times d_i$的箱子,最多两两组合成三种不同底面的箱子。
\item 要求下侧箱子底面的长和宽都\textbf{严格小于}上侧箱子,所以可知对于$h_i\times w_i\times d_i$的箱子,假设$h_i\leq w_i\leq d_i$,则三种不同底面的箱子\textbf{最多有其中两种}箱子可以叠在一个塔中。
\item 为了便于比较两个箱子长和宽,对于$h_i\leq w_i\leq d_i$的箱子,定义它的三种底面的\textbf{长}$\times$\textbf{宽}为元组$(a\times b)$分别为$\{(h_i,w_i),(w_i,d_i),(h_i,d_i)|i\in [1..n],h_i\leq w_i\leq d_i\}$。则\textbf{分别比较}$a_i<a_j,b_i<b_j$,即可得知底面$(a_i,b_i)$是否\textbf{严格小于}$(a_j,b_j)$。


\end{enumerate}

(二)由题意设计数据结构:

\textbf{记录3n个底面及高}:由上述分析,新建一个数组$bottom[1..3n]$,数组元素为箱子的\textbf{长宽高}元组$(a,b,c)$,对应$(bottom[i][0],bottom[i][1],bottom[i][2])$。其中元组的前两维作为箱子底面的长和宽,第三维作为箱子的高度。每个箱子分别用三个不同的面作为底面,底面的长和宽与其对应的高作为元组存进$bottom[1..3n]$数组,包括:$\{(h_i,w_i,d_i),(w_i,d_i,h_i),(h_i,d_i,w_i)|i\in [1..n],h_i\leq w_i\leq d_i\}$。对$bottom$以第一维作第一排序条件,第二维作第二排序条件,由小到大归并排序。

\textbf{记录状态}:$sum[i]$,表示$bottom[i]$的中保存的底面作为\textbf{最下方}箱子的底面时,塔的\textbf{最大总高度}。

\textbf{记录决策方案}:$upBox[i]$,当$bottom[i]$的面作为\textbf{最下方}的箱子的底面时,记其\textbf{上一个}箱子序号为$upBox[i]$,其底面为$bottom[upBox[i]]$中保存的底面。

\subsection{状态转移}
考虑每个箱子底面$bottom[i]$上方的最大塔高时,只需要考虑底面严格小于$(a_i,b_i)$的底面对应的最大塔高,再加上此底面对应的高$c_i$即可。由上述箱子的数据结构知,$bottom[j](j\in [1..i-1])$总满足$a_i\geq a_j$,$bottom[q](q\in [i+1..3n])$总满足$a_i\leq a_q$。故只需在$bottom[j](j\in [1..i-1])$中遍历,对于底面\textbf{严格小于}$(a_i,b_i)$的底面,找出其对应的\textbf{最大塔高}即可。

每个箱子$bottom[i]$为$(a_i,b_i,c_i)$,长为$a$和宽为$b$的面作为最下方的箱子的底面时,塔的最大总高度为箱子高度$c_i$与前序箱子$bottom[1..i-1]$中底面严格小于$(a_i,b_i)$的底面$sum[j](j\in [1..i-1])$对应的最大塔高之和$c_i+sum$。

故有如下递推式:

$sum[i]=bottom[i][2]+max\{sum[k]\}$,

其中$k\in [1,i-1]\ {\bf and}\ bottom[k][0]<bottom[i][0]\ {\bf and}\ bottom[k][1]<bottom[i][1]$

\subsection{记录决策方案}
记最大塔高对应的决策方案$boxes[boxes.length]$代表\textbf{自下而上的箱子序号},对应箱子为$bottom[boxes[i]]$。

记$upBox[i]$表示箱子最大高度$sum[i]$是由第$upBox[i]$个箱子的最大塔高$sum[upBox[i]]$加上当前的箱子高度$bottom[i][2]$转移而来的。

为了避免获取最大塔高时需要遍历查找最大值,记录当前最大塔高$maxHeight$和其最底部箱子序号$downBox$。

可以利用记录上侧箱子序号的$upBox[1..3n]$数组,与最大塔高对应的最底部箱子序号$boxes[1]=downBox$,得到倒数第二大的箱子序号为$boxes[2]=upBox[boxes[1]]$,倒数第三大的箱子序号为$upBox[boxes[2]],...$以此类推直到$upBox[boxes[boxes.length]]$为零,说明$boxes[boxes.length]$即为最顶部的箱子序号,从而迭代得到最底部箱子的上侧所有箱子序号,记为数组$boxes[1..boxes.length]$。

通过记录最大塔高对应的决策方案的所有\textbf{箱子序号}的数组$boxes[i](i\in [1..boxes.length])$,即可找到所有箱子对应的\textbf{底面和高度}$bottom[boxes[i]](i\in [1..boxes.length])$,代表最大塔高对应的决策方案。

\subsection{边界条件}
对于箱子$bottom[i]$,如果不存在前序箱子$bottom[1..i-1]$中底面严格小于$(a,b)$的底面,则箱子$bottom[i]$的最大塔高$sum[i]$为箱子高度$bottom[i][2]$。

此箱子上方没有箱子,故上方箱子序号$upBox[i]$记为$0$。

\subsection{目标状态}
塔的最高高度$max\{sum[i]\}(i\in [1..3n])$。

\subsection{复杂度分析}
总状态是$O(n)$级别的,每个状态的转移是$O(n)$的,所以总时间复杂度为$T(n)=O(n^2)$。

\textbf{此种方法复杂度为}$O(n^2)$\textbf{的伪代码见下页 Algorithm 5}

\subsection{方法二\ 使用树状数组优化时间复杂度为$O(log(n))$}
\textbf{(一)数据结构的设计:}

\textbf{1.记录3n个底面及高,与上述方法相同,保证长小于等于宽}:
新建一个数组$bottom[1..3n]$,数组元素为箱子的长宽高元组$(a,b,c)$,对应$(bottom[i][0],bottom[i][1],bottom[i][2])$。其中元组的前两维作为箱子底面的长和宽,第三维作为箱子的高度。每个箱子分别用三个不同的面作为底面,底面的长和宽与其对应的高作为元组存进$bottom[1..3n]$数组,包括:$\{(h_i,w_i,d_i),(w_i,d_i,h_i),(h_i,d_i,w_i)|i\in [1..n],h_i\leq w_i\leq d_i\}$。时间复杂度$O(n)$。

\textbf{2.对于箱子数组$bottom[1..3n]$,按照先使长递增再使宽递减归并排序,并去除相等的底面,得到$bottomSortByLength[1..m]$}:
对$bottom$以元组的第一维(长)作第一排序条件\textbf{升序},长相等时按元组的第二维(宽)作第二排序条件\textbf{降序},归并排序。时间复杂度$O(nlogn)$。

遍历此数组,如果存在\textbf{长宽均相等}的数组元素,只保留此底面与其\textbf{最大的高}。记\textbf{去重}后数组元素个数为$m$。时间复杂度$O(n)$。

此时得到长递增且长相等时宽递减的有序数组,简称其为\textbf{长递增宽递减}数组$bottomSortByLength[1..m]$。此数组有下列性质:对于任意底面$bottomSortByLength[k]$,长度严格小于此底面的箱子一定在长递增宽递减数组$bottomSortByLength[1..k-1]$中。

\textbf{3.对于长递增宽递减$bottomSortByLength[1..m]$数组与其对应下标index,按照先使宽递增再使长递减归并排序,得到另一个数组$bottomSortByWidth[1..m]$}:

数组$bottomSortByWidth[1..m]$的元素为四维元组$(a,b,c,index)$。归并排序时间复杂度$O(nlogn)$。

简称其为\textbf{宽递增长递减}数组$bottomSortByWidth[1..m]$。此数组有下列性质:对于任意底面宽递增长递减数组$bottomSortByWidth[k]$,宽度严格小于此底面的箱子一定在$bottomSortByWidth[1..k-1]$中。

\textbf{4.记录状态}:$sum[i]$,表示$bottomSortByLength[1..m]$的中保存的底面作为\textbf{最下方}箱子的底面时,塔的最大总高度。

\textbf{5.记录决策方案}:$upBox[i]$,当$bottomSortByLength[1..m]$的面作为最下方的箱子的底面时,记其\textbf{上一个}箱子序号为$upBox[i]$,底面为$bottomSortByLength[\ upBox[i]\ ]$中保存的底面。

\textbf{6.建立树状数组$tree[1..m]$}:使得对于任意下标$index$,可以以时间复杂度$O(logn)$查询并更新$sum[1..index]$中的最大值。


\textbf{(二)状态转移:}

遍历宽递增长递减数组$bottomSortByWidth[1..m]$,获取数组元素$bottomSortByWidth[k]$中存储的下标$index$,其中$index$对应的箱子底面为$bottomSortByLength[k]$,更新$sum[index]$的状态。

更新$sum[index]$的状态时,只需根据树状数组$tree[1..m]$,查询并更新$sum[1..k]$中的最大值。

遍历每个状态的时间复杂度$O(n)$,每个状态转移时,对于树状数组的查询并更新的时间复杂度为$O(nlogn)$。故状态转移的总复杂度$O(nlogn)$。

\textbf{(三)可以证明此种状态转移方式一定正确且最优:}

由(一)数据结构的2和3知:

1. 遍历到宽递增长递减数组的$bottomSortByWidth[k]$时,只有$bottomSortByWidth[1..k-1]$发生了更新。对于底面$(a_k,b_k)$,由宽递增长递减的排序条件可知,$bottomSortByWidth[1..k-1]$的宽一定小于或等于$b_k$;

2. 查询$sum[1..index]$中的最大值,相当于查询$bottomSortByLength[1..index]$中的最大塔高。对于下标$p\in [1..index]$,考虑底面$(a_p,b_p)$与底面$(a_k,b_k)$的关系:由数组$bottomSortByLength$长递增宽递减与预处理的去重,必有$a_p<a_k$,或者$a_p==a_k\ and\  b_p>b_k$。

\begin{itemize}
    \item  如果$a_p==a_k\ and\ b_p>b_k$:由数组$bottomSortByWidth$按宽递增遍历,更新$sum[k]$时,其$sum[p]$尚未被更新,$sum[p]=0$不影响最大值的查询。
    \item  如果$a_p<a_k$,且宽$b_p$大于$b_k$,则其$sum[p]$尚未被更新$sum[p]=0$不影响最大值的查询
    \item  如果$a_p<a_k$,则宽$b_p$等于$b_k$,由数组$bottomSortByWidth$按宽相等时长递减顺序遍历,更新$sum[k]$时,其$sum[p]$尚未被更新,$sum[p]=0$不影响最大值的查询。
    \item  如果$a_p<a_k$,且宽$b_p$小于$b_k$,则$sum[p]$作为更新$sum[k]$的值之一。
\end{itemize}

故由上述分析知,遍历到宽递增长递减数组的$bottomSortByWidth[k]$时,如果查询到长递增宽递减数组$bottomSortByLength[1..index]$中的最大非零塔高,则其底面\textbf{一定符合严格小于}的性质。

3. 对于未被遍历更新到的$bottomSortByWidth[k+1..n]$,宽比$b_k$大,不需要用其更新$sum[k]$。对于$bottomSortByLength[k+1..n]$,长比$a_k$大,不需要用其更新$sum[k]$。

4. 故得证。

\textbf{(四)状态转移方程及决策方案记录}

$$sum[index]=\max\limits_{k\in [1..index-1]}{sum[k]}+bottomSortByLength[index][2]$$

$\max\limits_{k\in [1..index-1]}{sum[k]}$通过树状数组$tree[1..m]$查询并维护。树状数组$tree[1..m][0]$记录最大值对应的$value$。$tree[1..m][1]$记录树状数组作为父节点被更新时,其\textbf{最底侧}箱子的序号,此值用于更新每个箱子得到最大塔高$sum[i]$时紧上面的箱子下标$upBox[i]$。


维护一个\textbf{全局最大}塔高$maxHeight$和对应的\textbf{最底部}箱子序号$downBox$。再根据$upBox[i]$记录最大值对应的\textbf{前序下标},与\textbf{方法一}中迭代寻找最优决策方案的方法相同。

\textbf{(五)时间复杂度分析}

预处理与归并排序的时间复杂度为$O(3nlog3n)$。

最多$3n$种状态,每个状态转移时采用树状数组,查询与维护最大塔高都是$O(log3n)$。故总时间复杂度为$T(n)=O(3nlog3n)=O(nlogn)$。

\textbf{此种方法复杂度为}$O(nlogn)$\textbf{的伪代码见下页 Algorithm 6-7}


\subsection{方法一:塔的最高高度$O(n^2)$的伪代码}
\begin{algorithm}[H]

\caption{方法一:塔的最高高度$O(n^2)$}
\LinesNumbered
\KwIn{代表每个箱子长宽高的三个数组$h[1.. n],w[1.. n],d[1.. n]$}
\KwOut{塔的最高高度}

\Function{$maxHeight(h[1.. n],w[1.. n],d[1.. n])$}{

	\tcp{\mbox{记录每个箱子的三个底面与其对应的高}}
	\textbf{let }$bottom[1..3n]$\textbf{ be new array with value }$\{(h_i,w_i,d_i),(w_i,d_i,h_i),(h_i,d_i,w_i)|i\in [1..n],h_i\leq w_i\leq d_i\}$\;
对$bottom[1..3n]$进行按第一维从小到大、第一维相等按第二维从小到大归并排序\;
	\tcp{\mbox{新建状态数组和记录每个箱子的紧上方箱子序号的数组}$sum[1..n],upBox[1..n]$}
	\textbf{let }$sum[1..n],upBox[1..n]$\textbf{ be new arrays with value 0}\;

	\tcp{$maxHeight,downBox$\mbox{记录当前最大塔高和对应底面的箱子序号}}
	$maxHeight\leftarrow 0,downBox\leftarrow 0$\;
	
	\For{$i\leftarrow{1}\ to\ n$}{
		
		\For{$j\leftarrow{1}\ to\ i-1$}{
			
			\If{$(bottom[i][0]\ >\ bottom[j][0])\ {\bf and}\ (bottom[i][1]\ >\ bottom[j][1])\ {\bf and}\ (sum[i]<sum[j])$}{
				$sum[i]\leftarrow sum[j],\ upBox[i]\leftarrow j$\;
			}
		}
		$sum[i]\leftarrow sum[i]+bottom[i][2]$\;
		\If{$sum[i]>maxHeight$}{
			$maxHeight\leftarrow sum[i],downBox\leftarrow i$\;
		}
	}

	\tcp{\mbox{最大塔高对应的决策方案}$boxes[boxes.length]$\mbox{代表自下而上的箱子序号}}
	\tcp{\mbox{序号}$boxes[i]$\mbox{对应的箱子为}$bottom[boxes[i]]$}
	\textbf{let }$boxes[1..3n]$\textbf{ be new array}\;
	
	$boxes.length\leftarrow 1$\;
	$boxes[boxes.length]\leftarrow downBox$\;
\tcp{\mbox{底侧箱子序号}$curBox$}
	$curBox\leftarrow boxes[boxes.length]$\;
\tcp{\mbox{底侧箱子的上一个箱子的序号}$lastBox$}
	$lastBox\leftarrow upBox[curBox]$\;

	\While{$lastBox[curBox]\neq 0$}{
		$boxes.length+=1$\;
		$boxes[boxes.length]=lastBox$\;
		$curBox\leftarrow boxes[boxes.length]$\;
		$lastBox\leftarrow upBox[curBox]$\;
	}
	
}

\end{algorithm}


\subsection{方法二:塔的最高高度$O(nlogn)$的伪代码}
\begin{algorithm}[H]
\caption{树状数组}

\LinesNumbered
\KwIn{代表每个箱子长宽高的三个数组$h[1.. n],w[1.. n],d[1.. n]$}
\KwOut{塔的最高高度}
\Function{$searchMaxAndUpdate(h,i)$}{
	$max\rightarrow 0, thisUpBox\rightarrow 0,find\_i\rightarrow i,update\_i\rightarrow i$\; 


	\tcp{$find\_i$\mbox{指向向下寻找最大值时经历的子节点,}$update\_i$\mbox{指向向上更新最大值时经历的父节点}}

	\While{$find\_i>0$}{
		\If{$max<tree[find\_i][0]$}{
			$max\leftarrow tree[find\_i][0]$\;
			$thisUpBox\leftarrow tree[find\_i][1]$\;
\tcp{\mbox{使用树状数组父节点作为最大塔高时,}$thisUpBox$\mbox{记录此最大塔高的最底层箱子序号}}
		}
$find\_i\leftarrow find\_i-lowbit(find\_i)$\;
	}

$max\leftarrow max+h$\;
$upBox[i]\leftarrow thisUpBox$\;
\tcp{\mbox{记录此被更新箱子的紧上方箱子序号}}
	\While{$update\_i\leq tree.length$}{
		\If{$max>tree[update\_i][0]$}{
			$tree[update\_i][0]\leftarrow max$\;
			$tree[update\_i][1]\leftarrow i$\;
\tcp{$tree[update\_i][1]$\mbox{记录:树状数组父节点被更新时,其最下方箱子的序号}}
\tcp{\mbox{用于后续使用此树状数组父节点作为最大塔高更新其他箱子时,作为其紧上方箱子序号}}
		}
$update_i\leftarrow update\_i+lowbit(update\_i)$\;
	}

}



\end{algorithm}
\textbf{接下页}

\begin{algorithm}[H]
\caption{方法二:塔的最高高度$O(nlogn)$}

\LinesNumbered
\KwIn{代表每个箱子长宽高的三个数组$h[1.. n],w[1.. n],d[1.. n]$}
\KwOut{塔的最高高度}

\Function{$maxHeight(h[1.. n],w[1.. n],d[1.. n])$}{

	\tcp{\mbox{数组,记录每个箱子的三个底面与其对应的高}}
	\textbf{let }$bottom[1..3n]$\textbf{ be new array value }$\{(h_i,w_i,d_i),(w_i,d_i,h_i),(h_i,d_i,w_i)|i\in [1..n],h_i\leq w_i\leq d_i\}$\;

	对$bottom[1..3n]$进行按第一维从小到大、第一维相等按第二维从大到小归并排序\;

	对$bottom[1..3n]$进行去重:遍历此数组,如果存在长宽均相等的数组元素,只保留此数组元素的第一维和第二维(底面)与其最大的高,记为数组$bottomSortByLength[1..m]$\;

	\textbf{let }$bottomSortByWidth[1..m]$\textbf{ be new array value copied from array }$bottomSortByLength[1..m]$\;

	令数组$bottomSortByWidth[1..m]$存储的元组增加一维\;
值为其第一次归并排序后的对应下标$index$\;
	对数组$bottomSortByWidth[1..m]$进行按第二维从小到大、第一维相等按第二维从大到小归并排序\;

	\tcp{\mbox{新建状态数组、树状数组}
$sum[1..m],tree[1..m][1..2]$}
	\tcp{\mbox{记录每个箱子的紧上方箱子序号的数组}
$upBox[1..m]$}
	\textbf{let }$sum[1..m],tree[1..m][1..2],upBox[1..m]$\textbf{ be new arrays with value 0}\;


	\tcp{$maxHeight,downBox$\mbox{记录当前最大塔高和对应底面的箱子序号}}
	$maxHeight\leftarrow 0,downBox\leftarrow 0$\;
	
	\For{$j\leftarrow{1}\ to\ n$}{
\tcp{\mbox{从}$bottomSortByLength[j][4]$\mbox{获取当前元素在数组}$bottomSortByLength$\mbox{中的下标}}
$i\leftarrow bottomSortByLength[j][4]$\;
		\tcp{$searchMaxAndUpdate()$\mbox{函数更新}$sum[i],\ upBox[i],\ tree[1..m][1..2]$}
		$searchMaxAndUpdate(bottomSortByLength[i][2],i)$\;
		
		\If{$sum[i]>maxHeight$}{
			$maxHeight\leftarrow sum[i],downBox\leftarrow i$\;
		}
	}

	\tcp{\mbox{最大塔高对应的决策方案}$boxes[boxes.length]$\mbox{代表自下而上的箱子序号}}
	\tcp{\mbox{序号}$boxes[i]$\mbox{对应的箱子为}$bottom[boxes[i]]$}
	\textbf{let }$boxes[1..3n]$\textbf{ be new array}\;
	
	$boxes.length\leftarrow 1$\;
	$boxes[boxes.length]\leftarrow downBox$\;
\tcp{\mbox{底侧箱子序号}$curBox$}
	$curBox\leftarrow boxes[boxes.length]$\;
\tcp{\mbox{底侧箱子的上一个箱子的序号}$lastBox$}
	$lastBox\leftarrow upBox[curBox]$\;

	\While{$lastBox[curBox]\neq 0$}{
		$boxes.length+=1$\;
		$boxes[boxes.length]=lastBox$\;
		$curBox\leftarrow boxes[boxes.length]$\;
		$lastBox\leftarrow upBox[curBox]$\;
	}
}
\end{algorithm}








\end{document}